\documentclass[a4paper]{article}

\usepackage{hyperref}
\usepackage[square,numbers,sort&compress,sectionbib]{natbib}

\newcommand{\code}[1]{\texttt{\small #1}}

\begin{document}

\title{Joern-0.2 Documentation}

\author{Fabian Yamaguchi}

\maketitle
\newpage
\tableofcontents
\newpage

\section{Overview}

Joern is a platform for robust analysis of C/C++ code developed by
\href{http://codeexploration.blogspot.de}{Fabian Yamaguchi} at the
\href{http://goesec.de}{Computer Security Group} of the University of
Goettingen. It generates \emph{code property graphs}, a novel graph
representation that exposes the code's syntax, control-flow, data-flow
and type information in a joint data structure. Code property graphs
are stored in a Neo4J graph database. This allows code to be mined
using search queries formulated in the graph traversal language
Gremlin.

In summary, Joern offers the following core features:

\begin{itemize}
  \item \textbf{Fuzzy Parsing.} Joern employs a fuzzy parser, allowing
    code to be imported even if a working build environment cannot be
    supplied.
    
  \item \textbf{Code Property Graphs.} Joern creates code property
    graphs from the fuzzy parser output and makes and stores them in a
    Neo4J graph database. For background information on code property
    graphs, we strongly encourage you to read our paper on the
    topic~\citep{YamGolArpRie14}.
    
  \item \textbf{Extensible Query Language.} Based on the graph
    traversal language Gremlin, Joern offers an extensible query language
    based on user-defined Gremlin steps that encode common traversals in
    the code property graph. These can be combined to create search
    queries easily.
    
\end{itemize}

This document provides the necessary documentation to enable you to
use Joern in your work. In particular, we cover its installation in
Section~\ref{sec:installation} and discuss how code can be imported
and retrieved from the database in
Section~\ref{sec:importAccess}. Section~\ref{sec:performance} offers
tips on performance tuning and is considered a must-read if you plan
to run queries on large code bases. Section~\ref{sec:dbOverview} gives
an overview of the database contents,, showing what kind of
information Joern makes available for your search queries. Finally,
Section~\ref{sec:examples} documents the available traversals for code
analysis and Section~\ref{sec:development} contains information for
developers.


\section{Installation}
\label{sec:installation}

Joern currently consists of the following components:

\begin{itemize}

\item \textbf{\href{https://github.com/fabsx00/joern/}{joern(-core)}}
  parses source code using a robust parser, creates code property graphs
  and finally, imports these graphs into a Neo4j graph database.
  
\item
  \textbf{\href{https://github.com/fabsx00/python-joern/}{python-joern}}
  is a (minimal) python interface to the Joern database. It offers a
  variety of utility traversals (so called \emph{steps}) for common
  operations on the code property graph (think of these are stored
  procedures).
  
\item \textbf{\href{https://github.com/fabsx00/joern-tools/}{joern-tools}}
  is a collection of command line tools employing python-joern to
  allow simple analysis tasks to be performed directly on the shell. 
  
\end{itemize}

Both python-joern and joern-tools are optional, however, installing
python-joern is highly recommended for easy access to the
database. While it is possible to access Neo4J from many other
languages, you will need to write some extra code to do so and
therefore, it is currently not recommended.

\subsection{System Requirements and Dependencies}

Joern is a Java Application and should work on systems offering a Java
virtual machine, e.g., Microsoft Windows, Mac OS X or GNU/Linux. We
have tested Joern on Arch Linux as well as Mac OS X Lion. If you plan
to work with large code bases such as the Linux Kernel, you should
have at least 30GB of free disk space to store the database and 8GB of
RAM to experience acceptable performance. In addition, the following
software should be installed:

\begin{itemize}
\item \textbf{\href{http://www.java.com/}{A Java Virtual Machine
      1.7.}} Joern is written in Java 7 and does not build with Java 6. It
  has been tested with OpenJDK-7 but should also work fine with Oracle's
  JVM.
  
\item
  \textbf{\href{http://www.neo4j.org/download/other\_versions}{Neo4J
      1.9.* Community Edition.}} The graph database Neo4J provides access to
  the imported code. Note, that Joern has not been tested with the 2.0
  branch of Neo4J.
  
\end{itemize}

\textbf{Build Dependencies.} A tarball containing all necessary
build-dependencies is available for download
\href{http://mlsec.org/joern/lib/lib.tar.gz}{here}. This contains
files from the following projects.

\begin{itemize}
  \item \href{http://www.antlr.org/}{The ANTLRv4 Parser Generator}
  \item \href{http://commons.apache.org/proper/commons-cli/}{Apache Commons CLI Command Line Parser 1.2}
  \item \href{http://www.neo4j.org/download/other\_versions}{Neo4J 1.9.* Community Edition}
  \item \href{http://ant.apache.org/}{The Apache Ant build tool} (tested with version 1.9.2)
\end{itemize}

The following sections offer a step-by-step guide to the installation
of Joern, including all of its dependencies.

\subsection{Building the Code}

\begin{enumerate}
  
  \item Begin by downloading the latest stable version of joern at
  
    \href{http://mlsec.org/joern/download.shtml}{http://mlsec.org/joern/download.shtml}. This
    will create the directory \code{joern} in your current working
    directory.
\begin{verbatim}
$ wget https://github.com/fabsx00/joern/archive/v0.2.tar.gz
$ tar xfzv v0.2.tar.gz
\end{verbatim}

  \item Change to the directory \code{joern/}. Next, download build
    dependencies at
    \href{http://mlsec.org/joern/lib/lib.tar.gz}{http://mlsec.org/joern/lib/lib.tar.gz}
    and extract the tarball. The JAR-files necessary to build Kern should
    now be located in \code{joern/lib/}.

\begin{verbatim}
$ cd joern
$ wget http://mlsec.org/joern/lib/lib.tar.gz
$ tar xfzv lib.tar.gz
\end{verbatim}
    
  \item Build the project by issuing the following command.
\begin{verbatim}
$ ant
\end{verbatim}

\item \textbf{Create symlinks (optional).} The executable JAR file
  will be located in \code{joern/bin/joern.jar}. Simply place 
this JAR file somewhere on your disk and you are done. If you are
using bash, you can optionally create the following alias in  your
\code{.bashrc}:
\begin{verbatim}
alias joern='java -jar $JOERN/bin/joern.jar'
\end{verbatim}  

where \code{\$JOERN} is the directory you installed Joern into.

\item \textbf{Build additional tools (optional).} Tools such as the
  \code{argumentTainter} can be built by issuing the following
  command.
  \begin{verbatim}
$ ant tools
\end{verbatim}

Upon successfully building the code, you can start importing C/C++
code you would like to analyze as outlined the next section.

\end{enumerate}

\section{Importing and Accessing Code}
\label{sec:importAccess}

\subsection{Importing Code}

Once joern has been installed, you can begin to import code into the
database by simply pointing \code{joern.jar} to the directory
containing the source code:

\begin{verbatim}
$ java -jar $JOERN/bin/joern.jar $CodeDirectory
\end{verbatim} or, if you want to ensure that the JVM has access to
your heap memory
\begin{verbatim}
$ java -Xmx$SIZEg -jar $JOERN/bin/joern.jar $CodeDirectory
\end{verbatim}

where \code{\$SIZE} is the maximum size of the Java Heap in GB. As an
example, you can import \code{\$JOERN/testCode}.

This will create a Neo4J database directory \code{.joernIndex} in your
current working directory. Note that if the directory already exists
and contains a Neo4J database, \code{joern.jar} will add the code to the
existing database. You can thus import additional code at any time. If
however, you want to create a new database, make sure to delete
\code{.joernIndex} prior to running \code{joern.jar}.


\subsection{Tainting Arguments}

Many times, an argument to a library function (e.g., the first
argument to \code{recv}) is tainted by the library function. There is
no way to statically determine this when the code of the library
function is not available. Also, Joern does not perform
inter-procedural taint-analysis and therefore, by default, symbols
passed to functions as arguments are considered \emph{used} but not
\emph{defined}.

To instruct Joern to consider arguments of a function to be tainted by
calls to that function, you can use the tool
\code{argumentTainter}. For example, by executing

\begin{verbatim}
java -jar ./bin/argumentTainter.jar recv 0
\end{verbatim}

from the Joern root directory, all first arguments to \code{recv} will
be considered tainted and dependency graphs will be recalculated
accordingly.

\subsection{Starting the Database Server}

It is possible to access the graph database directly from your scripts
by loading the database into memory on script startup. However, it is
highly recommended to access data via the Neo4J server instead. The
advantage of doing so is that the data is loaded only once for all
scripts you may want to execute allowing you to benefit from Neo4J's
caching for increased speed.

To install the neo4j server, download version \textbf{1.9.6} from:

\href{http://www.neo4j.org/download/other\_versions}{http://www.neo4j.org/download/other\_versions}

Once downloaded, unpack the archive into a directory of your choice,
which we will call \code{\$Neo4jDir} in the following.

Next, specificy the location of the database created by joern in your
Neo4J server configuration file in
\code{\$Neo4jDir/conf/neo4j-server.properties}:

\begin{verbatim}
# neo4j-server.properties
org.neo4j.server.database.location=/$path_to_index/.joernIndex/
\end{verbatim}

Second, start the database server by issuing the following command:

\begin{verbatim}
$ $Neo4jDir/bin/neo4j console
\end{verbatim}
	
The Neo4J server offers a web interface and a web-based API (REST API)
to explore and query the database. Once your database server has been
launched, point your browser to
\href{http://localhost:7474/}{http://localhost:7474/}. 

Next, visit
\href{http://localhost:7474/db/data/node/0}{http://localhost:7474/db/data/node/0}. This
is the \emph{reference node}, which is the root node of the graph
database. Starting from this node, the entire database contents can be
accessed. In particular, you can get an overview of all existing edge
types as well as the properties attached to nodes and edges.

Of course, in practice, you will not want to use your browser to query
the database. Instead, you can use python-joern to access the REST
API using Python as described in the following section.

\subsection{Accessing Code using python-joern}

Once code has been imported into a Neo4j database, it can be accessed
using a number of different interfaces and programming languages. One
of the simplest possibilities is to create a standalone Neo4J server
instance as described in the previous section and connect to this
server using python-joern, the python interface to joern.

To do so, install python-joern using pip:

\begin{verbatim}
$ sudo pip2 install git+git://github.com/fabsx00/python-joern.git
\end{verbatim}

Finally, run the following sample Python script, which prints all
assignments using a gremlin traversal:

\begin{verbatim}
# Hello World Script
from joern.all import JoernSteps

j = JoernSteps()
j.connectToDatabase()

query = 'queryNodeIndex("type:AssignmentExpr").code'

y = j.runGremlinQuery(query)
for x in y:
    print x
\end{verbatim}

It is highly recommended to test your installation on a small code
base first. The same is true for early attempts of creating search
queries, as erroneous queries will often run for a very long time on
large code bases, making a trial-and-error approach unfeasible.

\section{Performance Tuning}
\label{sec:performance}

\subsection{Optimizing Code Importing}

Joern uses the Neo4J Batch Inserter for code importing (see
\href{http://docs.neo4j.org/chunked/stable/batchinsert.html}{Chapter
  35 of the Neo4J documentation}). Therefore, the performance you will
experience mainly depends on the amount of heap memory you can make
available for the importer and how you assign it to the different
caches used by the Neo4J Batch Inserter. You can find a detailed
discussion of this topic
\href{https://github.com/jexp/batch-import}{here}.

By default, Joern will use a configuration based on the maximum size
of the Java heap. For sizes below 4GB, the following configuration is
used:

\begin{verbatim}
cache_type = none
use_memory_mapped_buffers = true
neostore.nodestore.db.mapped_memory = 200M
neostore.relationshipstore.db.mapped_memory = 2G
neostore.propertystore.db.mapped_memory = 200M
neostore.propertystore.db.strings.mapped_memory = 200M
neostore.propertystore.db.index.keys.mapped_memory = 5M
neostore.propertystore.db.index.mapped_memory = 5M
\end{verbatim}

The following configuration is used for heap-sizes larger than 4GB:

\begin{verbatim}
cache_type = none
use_memory_mapped_buffers = true
neostore.nodestore.db.mapped_memory = 1G
neostore.relationshipstore.db.mapped_memory = 3G
neostore.propertystore.db.mapped_memory = 1G
neostore.propertystore.db.strings.mapped_memory = 500M
neostore.propertystore.db.index.keys.mapped_memory = 5M
neostore.propertystore.db.index.mapped_memory = 5M
\end{verbatim}

The Neo4J Batch Inserter configuration is currently not
exported. If you are running Joern on a machine where these values
are too low, you can adjust the values in
\code{src/neo4j/batchinserter/ConfigurationGenerator.java}.

For the \code{argumentTainter}, the same default configurations are
used. The corresponding values reside in
\code{src/neo4j/readWriteDb/ConfigurationGenerator.java}.

\subsection{Optimizing Traversal Speed}

To experience acceptable performance, it is crucial to configure your
Neo4J server correctly. To achieve this, it is highly recommended to
review Chapter 22 of the Neo4J documentation on
\href{http://docs.neo4j.org/chunked/stable/embedded-configuration.html}{Configuration
  and Performance}. In particular, the following settings are
important to obtain good performance.

\begin{itemize}
  
  \item \textbf{Size of the Java heap.} Make sure the maximum size of
    the java heap is high enough to benefit from the amount of memory in
    your machine. One possibility to ensure this, is to add the
    \code{-Xmx\$SIZEg} flag to the variable \code{JAVA\_OPTS} in
    \code{\$Neo4JDir/bin/neo4j} where \code{Neo4JDir} is the directory of
    the Neo4J installation. You can also configure the maximum heap size
    globally by appending the \code{-Xmx} flag to the environment variable
    \code{\_JAVA\_OPTIONS}.
    
  \item \textbf{Maximum number of open file descriptors.} If, when
    starting the Neo4J server, you see the message

    \small{
\begin{verbatim}
WARNING: Max 1024 open files allowed, minimum of 40 000 recommended.
\end{verbatim}
    }
    you need to raise the maximum number of open file
    descriptors for the user running Neo4J (see the
    \href{http://docs.neo4j.org/chunked/stable/linux-performance-guide.html}{Neo4J
      Linux Performance Guide}).

  \item \textbf{Memory Mapped I/O Settings.} Performance of graph
    database traversals increases significantly when large parts of
    the graph database can be kept in RAM and do not have to be loaded
    from disk (see
    \href{http://docs.neo4j.org/chunked/stable/configuration-io-examples.html}{Neo4J
    Documentation}). For example, for a machine with 8GB RAM the
  following  \code{neo4j.conf} configuration has been tested to work
  well:
\begin{verbatim}
# conf/neo4j.conf
use_memory_mapped_buffers=true
cache_type=soft
neostore.nodestore.db.mapped_memory=500M
neostore.relationshipstore.db.mapped_memory=4G
neostore.propertystore.db.mapped_memory=1G
neostore.propertystore.db.strings.mapped_memory=1300M
neostore.propertystore.db.arrays.mapped_memory=130M
neostore.propertystore.db.index.keys.mapped_memory=200M
neostore.propertystore.db.index.mapped_memory=200M
keep_logical_logs=true
\end{verbatim}
    
  \end{itemize}

\section{Database Overview}
\label{sec:dbOverview}

In this section, we give an overview of the database layout created by
Joern, i.e., the nodes, properties and edges that make up the code
property graph. The code property graph created by Joern matches that
of the code property graph as described in the paper and merely
introduces some additional nodes and edges that have turned out to be
convenient in practice.

\subsection{Code Property Graphs}

For each function of the code base, the database stores a code
property graph consisting of the following nodes.

\begin{itemize}
  
  \item \textbf{Function nodes (type: Function).} A node for each
    function (i.e. procedure) is created. The function-node itself only
    holds the function name and signature, however, it can be used to
    obtain the respective Abstract Syntax Tree and Control Flow Graph of
    the function.
    
  \item \textbf{Abstract Syntax Tree Nodes (type:various).} Abstract
    syntax trees represent the syntactical structure of the code. They are
    the representation of choice when language constructs such as function
    calls, assignments or cast-expressions need to be located. Moreover,
    this hierarchical representation exposes how language constructs are
    composed to form larger constructs. For example, a statement may
    consist of an assignment expression, which itself consists of a left-
    and right value where the right value may contain a multiplicative
    expression (see
    \href{http://en.wikipedia.org/wiki/Abstract_syntax_tree}{Wikepedia:
      Abstract Syntax Tree} for more information). Abstract syntax
    tree nodes are connected to their child nodes with
    \code{IS\_AST\_PARENT\_OF} edges. Moreover, the corresponding
    function node is connected to the AST root node by a
    \code{IS\_FUNCTION\_OF\_AST} edge. 
    
  \item \textbf{Statement Nodes (type:various).} This is a sub-set of
    the Abstract Syntax Tree Nodes. Statement nodes are marked using
    the property \code{isCFGNode} with value \code{true}. Statement
    nodes are connected to other statement nodes via \code{FLOWS\_TO}
    and \code{REACHES} edges to indicate control and data flow
    respectively.
    
  \item \textbf{Symbol nodes (type:Symbol).} Data flow analysis is always
    performed with respect to a variable. Since our fuzzy parser needs
    to work even if declarations contained in header-files are missing,
    we will often encounter the situation where a \emph{symbol} is used,
    which has not previously been declared. We approach this problem by
    creating \emph{symbol} nodes for each identifier we encounter regardless
    of whether a declaration for this symbol is known or not. We also
    introduce symbols for postfix expressions such as \code{a->b} to allow us
    to track the use of fields of structures. Symbol nodes are connected
    to all statement nodes using the symbol by \code{USE}-edges and to all
    statement nodes assigning to the symbol (i.e., \emph{defining
      the symbol}) by \code{DEF}-edges.
    
\end{itemize}

Code property graphs of individual functions are linked in various
ways to allow transition from one function to another as discussed in
the next section.

\subsection{Global Code Structure}

In addition, to the nodes created for functions, the source file
hierarchy, as well as global type and variable declarations are
represented as follows.

\begin{itemize}
  
\item \textbf{File and Directory Nodes (type:File/Directory).} The
  directory hierarchy is exposed by creating a node for each file and
  directory and connecting these nodes using \code{IS\_PARENT\_DIR\_OF} and
  \code{IS\_FILE\_OF} edges. This \emph{source tree} allows code to
  be located by its location in the filesystem directory hierarchy,
  for example, this allows you to limit your analysis to functions
  contained in a specified sub-directory.
  
\item \textbf{Struct/Class declaration nodes (type: Class).} A
  Class-node is created for each structure/class identified and
  connected to file-nodes by \code{IS\_FILE\_OF} edges. The members
  of the class, i.e., attribute and method declarations are
  connected to class-nodes by \code{IS\_CLASS\_OF} edges.
  
\item \textbf{Variable declaration nodes (type: DeclStmt).} Finally,
  declarations of global variables are saved in declaration statement
  nodes and connected to the source file they are contained in using
  \code{IS\_FILE\_OF} edges.
  
\end{itemize}


\subsection{The Node Index}

In addition to the graphs stored in the Neo4J database, Joern makes an
Apache Lucene Index available that allows nodes to be quickly
retrieved based on their properties. This is particularly useful to
select start nodes for graph database traversals. For examples of node
index usage, refer to Section \ref{sec:startNode}.

\section{Querying the Database}
\label{sec:examples}

This chapter discusses how the database contents generated by Joern
can be queried to locate interesting code. We begin by reviewing the
basics of the graph traversal language Gremlin in Section
\ref{sec:gremlin} and proceed to discuss how to select start nodes
from the node index in Section \ref{sec:startNode}. The remainder of
this chapter deals with code retrieval based on syntax (Section
\ref{sec:ast} and \ref{sec:syntaxOnly}), taint-style queries in
Section \ref{sec:taint} and finally, traversals in the function symbol
graph in Section \ref{sec:symbolGraph}.

The user-defined traversals presented throughout this chapter are all
located in the directory \code{joernsteps} of python-joern. It may be
worth studying their implementation to understand how to design your
own custom steps for Gremlin.

\subsection{Gremlin Basics}
\label{sec:gremlin}

In this section, we will give a brief overview of the most basic
functionality offered by the graph traversal language Gremlin
developed by Marko A. Rodriguez. For detailed documentation of
language features, please refer to
\href{http://gremlindocs.com}{GremlinDocs.com} and the
\href{https://github.com/tinkerpop/gremlin/wiki}{Gremlin website}.

% There are currently two different query languages that can be used to
% retrieve information from Neo4J graph databases: Cypher and
% \href{https://github.com/tinkerpop/gremlin/wiki}{Gremlin}. While
% Cypher is simpler to learn than Gremlin, we found that it cannot
% compete with Gremlin in terms of flexibility. Unfortunately, many
% people find that Gremlin is rather difficult to learn as it follows a
% rather unusual programming paradigm, the flow-based programming
% paradigm. However, once mastered, the time invested is highly rewarded
% as Gremlin offers the ability to isolate  common graph traversals that
% can then be chained to formulate very complex retrievals that would be
% clumsy to express in languages such as Cypher. I hope that this
% section helps you to quickly get going with Gremlin and to appreciate
% its elegance and effectiveness.

Gremlin is a language designed to describe walks in property graphs. A
property graph is simply a graph where key-value pairs can be attached
to nodes and edges. (From a programmatic point of view, you can simply
think of it as a graph that has hash tables attached to nodes and
edges.)  In addition, each edge has a type, and that's all you need to
know about property graphs for now.


Graph traversals proceed in two steps to uncover to search a database
for sub-graphs of interest:

\begin{enumerate}
  
\item \textbf{Start node selection.} All traversals begin by
  selecting a set of nodes from the database that serve as starting
  points for walks in the graph.
  
\item \textbf{Walking the graph.} Starting at the nodes selected in
  the previous step, the traversal walks along the graph edges to
  reach adjacent nodes according to properties and types of nodes and
  edges. The final goal of the traversal is to determine all nodes
  that can be reached by the traversal. You can think of a graph
  traversal as a sub-graph description that must be fulfilled in order
  for a node to be returned.
  
\end{enumerate}

The simplest way to select start nodes is to perform a lookup based on
the unique node id as in the following query:
\begin{verbatim}
// Lookup node with given nodeId
g.v(nodeId)
\end{verbatim}

Walking the graph can now be achieved by attaching so called
\emph{Gremlin steps} to the start node. Each of these steps processes
all nodes returned by the previous step, similar to the way Unix
pipelines connect shell programs. While learning Gremlin, it can thus
be a good idea to think of the dot-operator as an alias for the unix
pipe operator ``\code{|}''. The following is a list of examples.

\begin{verbatim}
// Traverse to nodes connected to start node by outgoing edges
g.v(nodeId).out()

// Traverse to nodes two hops away.
g.v(nodeId).out().out()

// Traverse to nodes connected to start node by incoming edges
g.v(nodeId).in()

// All nodes connected by outgoing AST edges (filtering based
// on edge types)
g.v(nodeId).out(AST_EDGE)

// Filtering based on properties:
g.v(nodeId).out().filter{ it.type == typeOfInterest}

// Filtering based on edge properties
g.v.outE(AST_EDGE).filter{ it.propKey == propValue }.inV()
\end{verbatim}

The last two examples deserve some explanation as they both employ the
elementary step \code{filter} defined by the language
Gremlin. \code{filter} expects a so called \emph{closure}, an
anonymous function wrapped in curly brackets (see
\href{http://groovy.codehaus.org/Closures}{Groovy - Closures}). This
function is applied to each incoming node and is expected to return
\code{true} or \code{false} to indicate whether the node should pass
the filter operation or not. The first parameter given to a closure is
named \code{it} by convention and thus \code{it.type} denotes the
property \code{type} of the incoming node in the first of the two
examples.  In the second example, edges are handed to the filter
routine and thus \code{it.propKey} refers to the property
\code{propKey} of the incoming edge.

\subsection{Start Node Selection}
\label{sec:startNode}

In practice, the ids of interesting start nodes are rarely
known. Instead, start nodes are selected based on node properties, for
example, one may want to select all calls to function \code{memcpy} as
start nodes. To allow efficient start node selection based on node
properties, we keep an (Apache Lucene) node index (see
\href{http://docs.neo4j.org/chunked/stable/indexing.html}{Neo4J legacy
indices}). This index can be queried using Apache Lucene queries (see
\href{http://lucene.apache.org/core/2_9_4/queryparsersyntax.html}{Lucene
  Query Language} for details on the syntax). For example, to retrieve
all AST nodes representing callees with a name containing the
substring \code{cpy}, one may issue the following query:
\begin{verbatim}
queryNodeIndex("type:Callee AND name:*cpy*")
\end{verbatim}

The Gremlin step \code{queryNodeIndex} is defined in
\code{joernsteps/lookup.groovy} of python-joern. In addition to
\code{queryNodeIndex}, \code{lookup.groovy} defines various functions
for common lookups. The example just given could have also been
formulated as:
\begin{verbatim}
getCallsTo("*cpy*")
\end{verbatim}

Please do not hesitate to contribute short-hands for common lookup
operations to include in \code{joernsteps/lookup.groovy}.

% Possibly write something about mid-query lookups here

\subsection{Traversing Syntax Trees}
\label{sec:ast}

In the previous section, we outlined how nodes can be selected based
on their properties. As outline in Section \ref{sec:gremlin}, these
selected nodes can now be used as starting points for walks in the
property graph.

As an example, consider the task of finding all multiplications in
first arguments of calls to the function \code{malloc}. To solve this
problem, we can first determine all call expressions to \code{malloc}
and then traverse from the call to its first argument in the syntax
tree.  We then determine all multiplicative expressions that are child
nodes of the first argument.

In principle, all of these tasks could be solved using the elementary
Gremlin traversals presented in Section \ref{sec:gremlin}. However,
traversals can be greatly simplified by introducing the following
user-defined gremlin-steps (see \code{joernsteps/ast.py}).

\begin{verbatim}
// Traverse to parent nodes in the AST
parents()

// Traverse to child nodes in the AST
children()

// Traverse to i'th children in the AST
ithChildren()

// Traverse to enclosing statement node
statements()

// Traverse to all nodes of the AST
// rooted at the input node
astNodes()
\end{verbatim}
Additionally, \code{joernsteps/calls.groovy} introduces user-defined
steps for traversing calls, and in particular the step
\code{ithArguments} that traverses to i'th arguments of a given a call
node. Using these steps, the exemplary traversal for multiplicative 
expressions inside first arguments to \code{malloc} simply becomes:
\begin{verbatim}
getCallsTo('malloc').ithArguments('0')
.astNodes().filter{ it.type == 'MultiplicativeExpression'}
\end{verbatim}

\subsection{Syntax-Only Descriptions}
\label{sec:syntaxOnly}

The file \code{joernsteps/composition.groovy} offers a number of
elementary functions to combine other traversals and lookup
functions. These composition functions allow arbitrary syntax-only
descriptions to be constructed (see Modeling and Discovering
Vulnerabilities with Code Property Graphs~\citep{YamGolArpRie14}).

For example, to select all functions that contain a call to \code{foo}
AND a call to \code{bar}, lookup functions can simply be chained,
e.g.,
\begin{verbatim}
getCallsTo('foo').getCallsTo('bar')
\end{verbatim}
returns functions calling both \code{foo} and \code{bar}. Similarly,
functions calling \code{foo} OR \code{bar} can be selected as follows:
\begin{verbatim}
OR( getCallsTo('foo'), getCallsTo('bar') )
\end{verbatim}

Finally, the \code{not}-traversals allows all nodes to be selected
that do NOT match a traversal. For example, to select all functions
calling \code{foo} but not \code{bar}, use the following traversal:
\begin{verbatim}
getCallsTo('foo').not{ getCallsTo('bar') }
\end{verbatim}

\subsection{Traversing the Symbol Graph}
\label{sec:symbolGraph}

As outlined in Section \ref{sec:dbOverview}, the symbols used and
defined by statements are made explicit in the graph database by
adding symbol nodes to functions (see Appendix D of
~\citep{YamGolArpRie14}). We provide utility traversals to make use of
this in order to determine symbols defining variables, and thus simple
access to types used by statements and expressions. In particular, the
file \code{joernsteps/symbolGraph.groovy} contains the following
steps:

\begin{verbatim}
// traverse from statement to the symbols it uses
uses()

// traverse from statement to the symbols it defines
defines()

// traverse from statement to the definitions
// that it is affected by (assignments and
// declarations)
definitions()
\end{verbatim}

As an example, consider the task of finding all third arguments to
\code{memcpy} that are defined as parameters of a function. This can
be achieved using the traversal
\begin{verbatim}
getArguments('memcpy', '2').definitions()
.filter{it.type == TYPE_PARAMETER}
\end{verbatim}
where \code{getArguments} is a lookup-function defined in
\code{joernsteps/lookup.py}.

As a second example, we can traverse to all functions that use a
symbol named \code{len} in a third argument to \code{memcpy} that is
not used by any condition in the function, and hence, may not be
checked.

\begin{verbatim}
getArguments('memcpy', '2').uses()
.filter{it.code == 'len'}
.filter{
  it.in('USES')
 .filter{it.type == 'Condition'}.toList() == []
}
\end{verbatim}

This example also shows that traversals can be performed inside
filter-expressions and that at any point, a list of nodes that the
traversal reaches can be obtained using the function \code{toList}
defined on all Gremlin steps.

\subsection{Taint-Style Descriptions}
\label{sec:taint}

The last example already gave a taste of the power you get when you
can actually track where identifiers are used and defined. However,
using only the augmented function symbol graph, you cannot be sure the
definitions made by one statement actually \emph{reach} another
statement. To ensure this, the classical \emph{reaching definitions}
problem needs to be solved. In addition, you cannot track whether
variables are sanitized on the way from a definition to a statement.

Fortunately, Joern allows you to solve both problems using the
traversal \code{unsanitized}. As an example, consider the case where
you want to find all functions where a third argument to \code{memcpy}
is named \code{len} and is passed as a parameter to the function and a
control flow path exists satisfying the following two conditions:

\begin{itemize}
  \item The variable \code{len} is not re-defined on the way.
  \item The variable is not used inside a relational or equality
    expression on the way, i.e., its numerical value is not
    ``checked'' against some other variable.
\end{itemize}

You can use the following traversal to achieve this:

\begin{verbatim}
getArguments('memcpy', '2')
.sideEffect{ paramName = '.*len.*' }
.filter{ it.code.matches(paramName) }
.unsanitized{ it.isCheck( paramName ) }
.params( paramName )
\end{verbatim}

where \code{isCheck} is a traversal defined in
\code{joerntools/misc.groovy} to check if a symbol occurs inside an
equality or relational expression and \code{params} traverses to
parameters matching its first parameter.

Note, that in the above example, we are using a regular
expression to determine arguments containing the sub-string \code{len}
and that one may want to be a little more exact here. Also, we use the
Gremlin step \code{sideEffect} to save the regular expression in a
variable, simply so that we do not have to re-type the regular
expression over and over.

\section{Development}
\label{sec:development}

\subsection{Accessing the GIT Repository}

We use the revision control system git to develop Joern. If you want
to participate in development or test the development version, you can
clone the git repository by issuing the following command:

\begin{verbatim}
  git clone https://github.com/fabsx00/joern.git
\end{verbatim}

Optionally, change to the branch of interest. For example, to test the
development version, issue the following:

\begin{verbatim}
  git checkout dev
\end{verbatim}

If you want to report issues or suggest new features, please do so via
\href{https://github.com/fabsx00/joern}{github}. For fixes, please
fork the repository and issue a pull request or alternatively send a
diff to the developers by mail.

\subsection{Modifying Grammar Definitions}

When building Joern, pre-generated versions of the parsers will be
used by default. This is fine in most cases, however, if you want to
make changes to the grammar definition files, you will need to
regenerate parsers using the antlr4 tool. For this purpose, it is
highly recommended to use the optimized version of ANTLR4 to gain
maximum performance.

To build the optimized version of ANTLR4, do the following:
\begin{verbatim}
$ git clone https://github.com/sharwell/antlr4/
$ cd antlr4
$ mvn -N install
$ mvn -DskipTests=true -Dgpg.skip=true -Psonatype-oss-release
  -Djava6.home=$PATH_TO_JRE install
\end{verbatim}

If the last step gives you an error, try building without
\code{-Psonatype-oss-release}.

\begin{verbatim}
$ mvn -DskipTests=true -Dgpg.skip=true -Djava6.home=$PATH_TO_JRE install
\end{verbatim}

Next, copy the antlr4 tool and runtime to the following locations:

\begin{verbatim}
$ cp tool/target/antlr4-$VERSION-complete.jar $JOERN/
$ cp runtime/Java/target/antlr4-runtime-$VERSION-SNAPSHOT.jar $JOERN/lib
\end{verbatim}

where \$JOERN is the directory containing the \$JOERN installation.

Parsers can then be regenerated by executing the script
\code{\$JOERN/genParsers.sh}.


% \section{Reference}

% \subsection{Node Types}

% \subsection{Edge Types}

% \textbf{Directory hierarchy $\rightarrow$ functions and declarations.}
% \begin{itemize}
% \item \textbf{\code{IS\_PARENT\_DIR\_OF}:} connects directories to the files and
%   sub-directories they contain.
  
% \item \textbf{\code{IS\_FILE\_OF}:} connects files to the functions
%   and variable declarations they contain.  
% \end{itemize}
% \textbf{Functions $\rightarrow$ ASTs and CFGs.}

% \begin{itemize}
%   \item \textbf{\code{IS\_FUNCTION\_OF\_AST:}} connects functions to
%     their abstract syntax trees.
    
%   \item \textbf{\code{IS\_FUNCTION\_OF\_CFG:}} connects functions to
%     their CFGs.
% \end{itemize}
% \textbf{AST/CFG edges.}
% \begin{itemize}
%   \item \textbf{\code{IS\_AST\_PARENT:}} connects parent AST nodes to their children.
    
%   \item \textbf{\code{FLOWS\_TO:}} connects statement nodes to
%     successors in the control flow graph, i.e. statements possibly
%     executed right after the current statement.    
% \end{itemize}
% \textbf{Edges for type analysis.}
% \begin{itemize}
%   \item \textbf{\code{IS\_CLASS\_OF:}} connects structures/classes to
%     their members.
%   \item \textbf{\code{DECLARES:}} connects declaration statements to
%     the declarations they contain, e.g., \code{int x, y} contains the
%     two declarations \code{int x} and \code{int y}.

% \end{itemize}
% \textbf{Edges for data flow analysis.}
% \begin{itemize}
% \item \textbf{\code{DEF:}} connects symbol nodes to the AST nodes
%   where the symbol is \emph{defined}, i.e., where it is first declared
%   or assigned to.
% \item \textbf{\code{USE:}} connects symbol nodes to the AST nodes
%   where the symbol is used.
% \item \textbf{\code{REACHES:}} connects statement nodes to the
%   statements reached via data-flow. REACHES edges are labeled by
%   variables.
% \end{itemize}

\bibliography{joern-docs.bib}
\bibliographystyle{abbrvnat}

\end{document}
